\documentclass[a4paper]{article}
\usepackage[utf8]{inputenc}
\usepackage[T1]{fontenc}
\usepackage{hyperref}
\usepackage{graphicx}
\usepackage{float}
\usepackage{svg}
\usepackage{caption}
\captionsetup{font=footnotesize}
\captionsetup{width=0.8\textwidth}
\usepackage[french]{babel}
\author{Régis Portalez\\
\large \textit{COMUA\textsuperscript{TM}}}
\date{\today}
\title{Louise\\Notice}
\renewcommand{\contentsname}{Table des matières}
\begin{document}

\maketitle

\begin{figure}[H]
    \centering
    \includegraphics*[width=0.8\textwidth]{../images/prototype}
    \caption{Prototype monté et installé}
    \label{fig:prototype}
\end{figure}
Louise est un tour de potier professionnel bâti essentiellement en acier, inox et aluminium. Pour en
assurer la plus grande durée de vie possible, les travailleurs l’utilisant doivent en prendre grand soin
et respecter les procédures détaillées dans ce document.
\newpage
\tableofcontents
\newpage
\section{Notice}
\subsection{Mise en place}
Louise doit être placée dans un atelier sec et à température située entre 5 et 30 degrés. Elle doit
être posée sur un sol plan et solide afin de limiter les vibrations. Avant la première utilisation, on
doit s’assurer qu’elle est à la fois à la hauteur désirée par le travailleur et parfaitement de niveau.
Pour ce faire, poser un niveau sur la girelle et jouer sur les vis des quatre pieds jusqu’à ce que la
girelle soit à la fois à la bonne hauteur et de niveau selon deux axes perpendiculaires.
Attention~: la machine doit impérativement être posée sur ses pieds, même à hauteur minimale, sans
quoi le moteur et le variateur ne seront pas ventilés, ce qui va réduire leur durée de vie et provoquer
des pannes.
\subsection{Branchement}
\begin{itemize}
\item Brancher la pédale de commande sur le bâti du tour (attention à respecter le sens des fiches) ;
\item Brancher la prise sur une prise de courant dûment protégée et raccordée à la terre ;
\item Attendre que la lumière verte s’allume près des boutons latéraux ;
\item Démarrer en appuyant sur le bouton vert ;
\item Vérifier en accélérant légèrement que la girelle tourne dans le sens désiré par le travailleur (le cas
échéant, arrêter la rotation et inverser le sens – voir ci-dessous) ;
\item La machine est prête à l’utilisation.
\end{itemize}

\subsection{Boutons de commande}
\begin{itemize}
\item Démarrage du variateur~: bouton vert (s’assurer que la pédale est en position arrêt – enfoncée à
fond côté talon).
\item Arrêt~: bouton rouge. Avant de débrancher la machine, il est fortement recommandé de patienter
que le variateur s’éteigne. Cela peut mettre jusqu’à dix secondes ;
\item Arrêt d’urgence~: mettre la vitesse de rotation à zéro en appuyant à fond sur la pédale côté talon. La
machine s’arrête alors de tourner en une demi seconde.
\item Inversion sens de rotation~: une fois la girelle immobile, faire effectuer un quart de tour (dans un
sens ou dans l’autre) au bouton de commande latéral. La girelle change alors de sens de rotation.
Attention~: ne jamais effectuer cette opération avec le moteur en rotation. Cela peut détériorer le
moteur ou le variateur et réduire leur durée de vie.
\end{itemize}

\subsection{Opérations de maintenance régulières}
Les carters et le bac (selon les modèles) de Louise sont en acier. Malgré la peinture, cette matière est sensible à la
corrosion, notamment quand la terre travaillée est elle-même acide ou corrosive.
L’utilisateur doit donc soigneusement nettoyer et sécher bac, girelle et broche après chaque journée
de travail. De même, il est fortement recommandé de stopper la machine le soir, d’attendre quelques
secondes et de la débrancher.

Tous les mois, démonter bac et girelle. Nettoyer la partie femelle de la girelle (sur le dessous, la
bague acier ou inox qui s’emmanche dans la broche), la dépoussiérer et la graisser soigneusement
avec une graisse industrielle visqueuse quelconque, trouvable dans n’importe quel magasin de
bricolage. Faire de même sur la broche, nettoyer, dépoussiérer et graisser la broche et le joint SPI
assurant l’étanchéité du carter principal. Enlever les excès de graisse avec un chiffon en tissu.

La girelle est en aluminium massif. C’est une matière faiblement oxydable, mais particulièrement
tendre. Malgré son épaisseur et sa robustesse, les travailleurs doivent prendre grand soin de ne pas
la rayer et la protéger des chocs. Le cas échéant, elle pourra ou non être reprise en atelier pour la
remettre en état et éviter sa mise au rebut.

\subsection{Opérations de maintenance rares et déconseillées}
Les opérations suivantes valent nullité de la garantie. Si elles s'avèrent absolument necéssaires, parce que
votre machine a subi un choc ou un sinistre grave, veuillez nous contacter avant de les effectuer. 

Le variateur de fréquence est accessible en démontant la trappe arrière du carter principal. Cette
opération doit être effectuée une fois la machine débranchée et le voyant vert éteint. On peut alors
accéder aux réglages du variateur pour le paramétrer à sa convenance en suivant attentivement la
notice de celui-ci, fournie avec la machine. Cette opération est potentiellement destructrice pour la
machine et ne doit être effectuée qu’en cas d’extrême nécessité et par une personne qualifiée.

Le carter principal se démonte également intégralement grâce aux vis présentes tout autour du bâti.
De même, cette opération doit être effectuée hors tension, c’est-à-dire une fois la machine stoppée,
débranchée et que le voyant vert est éteint. Le réglage de l’inclinaison de la broche se fait alors en
desserrant les vis attachant le réducteur au bâti. Une fois le parallélisme souhaité obtenu, il est
recommandé de bloquer les vis une par une en utilisant une colle type loctite afin d’éviter qu’un jeu
s’installe avec le temps. De même, cette opération est potentiellement destructrice pour la machine
et ne doit être effectuée qu’en cas d’extrême nécessité et par une personne qualifiée. 

\section{Plans et construction}
\subsection{Bâti}
L'ensemble du bâti doit être monté et soudé ensemble à l'exception des plaques de support du réducteur, à monter à la fin. 
Les cotes de découpe doivent être respectées dans le millimètre sous peine de devoir récupérer de nombreux équerrages. 
Pensez à ébarber, blanchir et dégraisser les zones de soudage afin d'assurer une pénétration optimale des cordons de soudure.
\begin{figure}[H]
    \centering
    \includegraphics*[width=0.8\textwidth]{../plans/bati}
    \caption{bati}
    \label{fig:bati}
\end{figure}

\subsubsection{Base}
\begin{figure}[H]
    \centering
    \includegraphics*[width=0.8\textwidth]{../plans/base}
    \caption{base}
    \label{fig:base}
\end{figure}
\subsubsection{Joints bases}
\begin{figure}[H]
    \centering
    \includegraphics*[width=0.8\textwidth]{../plans/joint-base}
    \caption{joints}
    \label{fig:joints}
\end{figure}
\subsubsection{Tubes bâti}
\begin{figure}[H]
    \centering
    \includegraphics*[width=0.8\textwidth]{../plans/tube-bati-haut}
    \caption{tube bâti haut}
    \label{fig:tube-bati-haut}
\end{figure}
\begin{figure}[H]
    \centering
    \includegraphics*[width=0.8\textwidth]{../plans/tube-bati-vertical}
    \caption{tube bâti vertical}
    \label{fig:tube-bati-vertical}
\end{figure}
\begin{figure}[H]
    \centering
    \includegraphics*[width=0.8\textwidth]{../plans/tube-bati-moteur}
    \caption{tube bâti moteur}
    \label{fig:tube-bati-moteur}
\end{figure}

\subsection{Motorisation}
Louise est équipée par défaut d'un moteur triphasé de 550W (1450 rpm), doublé d'un réducteur roue et vis d'un rapport 
de 1:5. Le moteur est monté sur pattes B4 et le réducteur doit avoir des vis d'accroches espacées horizontalement de 80mm
et verticalement de 70mm. Le cas échéant, adapter les pièces correspondantes. 

\subsubsection{Broche}
La broche doit être réalisée en inox. Les diamètres de 25mm sont en H7 (+0/-0.02mm). Le diamètre intermédiaire importe peu mais ne doit pas
dépasser 35mm. La rainure de clavette ne doit pas nécessairement s'étirer sur toute la longueur du grand diamètre mais faire au moins 40mm. 
\begin{figure}[H]
    \centering
    \includegraphics*[width=0.8\textwidth]{../plans/broche}
    \caption{broche}
    \label{fig:base}
\end{figure}
\subsubsection{Plaques moteur}
Le réducteur est fixé à ces plaques par 6 ou 8 vis de 8mm équipées de rondelles Grower et d'écrous Nylstop.
\begin{figure}[H]
    \centering
    \includegraphics*[width=0.8\textwidth]{../plans/plaques}
    \caption{plaques support réducteur}
    \label{fig:plaques}
\end{figure}
Une fois tout le bâti construit, on peut attacher les plaques au réducteur, positionner ces plaques sur le bâti de manière à ce que la broche
tourne avec une concentricité inférieure à 5 centièmes de millimètres (placer un comparateur sur le bâti et mesurer les oscillations de la broche). 
Ceci effectué, pointer, contrôler la concentricité, souder. 


\end{document}